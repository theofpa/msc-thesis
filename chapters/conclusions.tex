\section{Conclusions}

Grid monitoring is an important factor when working with Grid Systems. Reports extracted from monitoring systems, support decisions for capacity management, and proove that a system meets requirements needed for Service Level Aggrements.

Information Service is core technology that used in Grid Computing. It has evolved in parallel with Middleware. Current version of the Monitoring and Discovery Service standard has reached the MDS4, introducing the use of Web Services. MDS2 was based in LDAP, which is still used by some systems to discover services.

BDII Information Service is great for use in site level environment. Site monitoring needs Nagios and Ganglia for a few hundreds of nodes. WSRF caching feature, Indexer and Aggregator Framework scales better and may be used for regional and top level monitoring.



Conclusions should be based on an in depth critical analysis of the information presented in the dissertation and should be related to the objectives stated in the introduction.

do not simply summarize the dissertation

do not recapitulate the analysis or discussion

do not introduce new ideas

identify specific points that have been clarified or discovered, and specific actions to be taken

identify specific additional investigation that is required (and why)

It is important to remember that conclusions should only be drawn on the basis of the information presented in the dissertation. Generalized conclusions without supporting evidence are to be
discouraged.

\section{Further Work}
Identify specific additional investigation that is required to be carried out.

For testing purposes some URLs of other 

/opt/globus/etc/wsrf..,../hierarchy.xml maybe in an "aggregation" section

\subsection{Storage element performance monitoring}
Storage Element performance monitoring, IOPS. Cases using IBM TotalStorage, NetApp OnTap and EMC Navisphere
FC vs CFS \cite{brzezniak2008analysis}
hadoop for CMS \cite{hadoop}

