\section{Tables and Plots}
\subsection{Unix stuff}

As described in subsection \nameref{subsec:metrics} in the previous chapter, linux provides through the {\bf proc pseudo-filesystem} a simple file interface to the metrics taken from the scheduler of processes that are queued in the processor. The three metrics about cpu load on 1, 5 and 15 minutes average is displayed as follows:

\begin{verbatim}
[root@gr03 ~]# cat /proc/loadavg 
2.29 0.73 0.32 1/230 3584
\end{verbatim}

which may also be displayed using the {\bf uptime command}:

\begin{verbatim}
[root@gr03 ~]# uptime
 00:01:20 up  1:41,  3 users,  load average: 2.29, 0.73, 0.32
\end{verbatim}

Looking in the Linux kernel source, there is the CALC\_LOAD macro command which takes the options that have been discused and returns the result of the metric. The definition of the macro can be seen in file {\bf /usr/src/kernels/`uname -r`/include/linux/sched.h}:

\begin{lstlisting}[caption=Linux kernel CALC\_LOAD macro]
extern unsigned long avenrun[];         /* Load averages */
extern void get_avenrun(unsigned long *loads, 
                        unsigned long offset, int shift);

#define FSHIFT    11          /* nr of bits of precision */
#define FIXED_1   (1<<FSHIFT) /* 1.0 as fixed-point */
#define LOAD_FREQ (5*HZ+1)    /* 5 sec intervals */
#define EXP_1     1884        /* 1/exp(5sec/1min) as fixed-point */
#define EXP_5     2014        /* 1/exp(5sec/5min) */
#define EXP_15    2037        /* 1/exp(5sec/15min) */

#define CALC_LOAD(load,exp,n) \
        load *= exp; \
        load += n*(FIXED_1-exp); \
        load >>= FSHIFT;

extern unsigned long total_forks;
extern int nr_threads;
DECLARE_PER_CPU(unsigned long, process_counts);
extern int nr_processes(void);
extern unsigned long nr_running(void);
extern unsigned long nr_uninterruptible(void);
extern unsigned long nr_iowait(void);
extern unsigned long nr_iowait_cpu(int cpu);
extern unsigned long this_cpu_load(void);
\end{lstlisting}

\newpage

\subsection{Ganglia}

\subsubsection{Installation and configuration}

\begin{lstlisting}[language=bash,caption=Gmetad installation]
# yum install rrdtool perl-rrdtool rrdtool-devel apr-devel libconfuse libconfuse-devel expat expat-devel pcre pcre-devel
# GANGLIA_ACK_SYSCONFDIR=1 ./configure --with-gmetad
# make
# make install
# mkdir -p /var/lib/ganglia/rrds
# chown -R nobody /var/lib/ganglia
\end{lstlisting}

\begin{lstlisting}[language=bash,caption=Gmond installation]
# yum install apr-devel libconfuse libconfuse-devel expat expat-devel pcre pcre-devel
# GANGLIA_ACK_SYSCONFDIR=1 ./configure
# make
# make install
# iptables -A RH-Firewall-1-INPUT -m state --state NEW -m tcp -p tcp --dport 8649 -j ACCEPT
\end{lstlisting}

configure with multicast or unicast. jitter problem environments may use unicast, as long as amazon ec2 environments that are not support multicast.
\newpage

\subsubsection{Transport and sample}
Gmond code uses the ganglia libmetrics library which in case of Linux operating system parses the $/proc/loadavg$ pseudo-file to get linux kernel calculated system load average.

\begin{lstlisting}[language=C,caption=libmetrics code to get load average]
timely_file proc_loadavg = { {0,0} , 5., "/proc/loadavg" };
/* ... */
g_val_t
load_one_func ( void )
{
   g_val_t val;
   val.f = strtod( update_file(&proc_loadavg), (char **)NULL);
   return val;
}
\end{lstlisting}
code from gmond client

example output 

When gmond starts, it listens on port 8649/TCP by default, to accept TCP connections and throw XML report for the whole cluster. It also binds to the multicast address on port 8649/UDP to get other hosts messages for metrics changes, and also multicast its own metrics.
\begin{lstlisting}[language=bash,caption=Gmond networking]
[root@gr01 ~]# lsof -i 4 -a -p `pidof gmond`
COMMAND   PID   USER   FD   TYPE DEVICE SIZE NODE NAME
gmond   11900 nobody    4u  IPv4  33699       UDP 239.2.11.71:8649 
gmond   11900 nobody    5u  IPv4  33701       TCP *:8649 (LISTEN)
gmond   11900 nobody    6u  IPv4  33703       UDP gr01.oslab.teipir.gr:39991->239.2.11.71:8649 
\end{lstlisting}
port and XML staff

\begin{lstlisting}[language=XML,caption=Gmond XML cluster report]
<?xml version="1.0" encoding="ISO-8859-1" standalone="yes"?>
<GANGLIA_XML VERSION="3.1.7" SOURCE="gmond">
	<CLUSTER NAME="RDLAB" LOCALTIME="1297198943" OWNER="TEIPIR" LATLONG="unspecified" URL="unspecified">
		<HOST NAME="gr02.oslab.teipir.gr" IP="10.0.0.32" REPORTED="1297198934" TN="8" TMAX="20" DMAX="0" LOCATION="unspecified" GMOND_STARTED="1296569542">
			<METRIC NAME="load_one" VAL="0.01" TYPE="float" UNITS=" " TN="50" TMAX="70" DMAX="0" SLOPE="both">
			<EXTRA_DATA>
			<EXTRA_ELEMENT NAME="GROUP" VAL="load"/>
			<EXTRA_ELEMENT NAME="DESC" VAL="One minute load average"/>
			<EXTRA_ELEMENT NAME="TITLE" VAL="One Minute Load Average"/>
			</EXTRA_DATA>
			</METRIC>
			...
		</HOST>
		...
	</CLUSTER>
</GANGLIA_XML>
\end{lstlisting}
multicast examples
\newpage

\subsubsection{Gmetad}
\newpage

XDR output

\begin{lstlisting}[language=bash,caption=XDR sample]
[root@gr01 ~]# tcpdump -A -i eth2 dst host 239.2.11.71
22:38:26.062266 IP gr01.oslab.teipir.gr.39991 > 239.2.11.71.8649: UDP, length 56
E..T..@..........G.7!..@.X........gr01.oslab.teipir.gr....load_one........%.2f..
\end{lstlisting}

gstat -al1

\begin{lstlisting}[language=bash,caption=Gstat output]
[root@gr01 ~]# gstat -al1
gr03.oslab.teipir.gr     2 (    0/   87) [  0.00,  0.00,  0.00] [   0.0,   0.0,   0.0,  99.9,   0.1] OFF
gr01.oslab.teipir.gr     1 (    0/   75) [  0.00,  0.00,  0.00] [   0.0,   0.0,   0.0,  99.9,   0.0] OFF
gr02.oslab.teipir.gr     1 (    0/   99) [  0.00,  0.00,  0.00] [   0.0,   0.0,   0.1,  99.9,   0.0] OFF
\end{lstlisting}
\newpage

\subsection{Visualization}
Drill down and levels of visualization
\begin{enumerate}
  \item local resource
  \item site
  \item regional
  \item grid
\end{enumerate}

ganglia GUI

rrdtool
\newpage

\section{Methods of Presentation}


\subsection{WSRF}
\begin{verbatim}
/opt/globus/bin/wsrf-query \
-s https://osweb.teipir.gr:8443/wsrf/services/DefaultIndexService \
"//*[local-name()='Host']"
\end{verbatim}


\begin{lstlisting}[language=XML,caption=WSRF query output]
 <ns1:GLUECE xmlns:ns1="http://mds.globus.org/glue/ce/1.1">
  <ns1:Cluster ns1:Name="OSLAB" ns1:UniqueID="OSLAB">
   <ns1:SubCluster ns1:Name="main" ns1:UniqueID="main">
    <ns1:Host ns1:Name="gr03.oslab.teipir.gr" 
    ns1:UniqueID="gr03.oslab.teipir.gr" 
    xmlns:ns1="http://mds.globus.org/glue/ce/1.1">
     <ns1:Processor ns1:CacheL1="0" ns1:CacheL1D="0" 
     ns1:CacheL1I="0" ns1:CacheL2="0" ns1:ClockSpeed="2392" 
     ns1:InstructionSet="x86"/>
     <ns1:MainMemory ns1:RAMAvailable="299" ns1:RAMSize="1010" 
     ns1:VirtualAvailable="2403" ns1:VirtualSize="3132"/>
     <ns1:OperatingSystem ns1:Name="Linux"
     ns1:Release="2.6.18-194.26.1.el5"/>
     <ns1:Architecture ns1:SMPSize="2"/>
     <ns1:FileSystem ns1:AvailableSpace="201850" 
     ns1:Name="entire-system" ns1:ReadOnly="false"
     ns1:Root="/" ns1:Size="214584"/>
     <ns1:NetworkAdapter ns1:IPAddress="10.0.0.33" 
     ns1:InboundIP="true" ns1:MTU="0" 
     ns1:Name="gr03.oslab.teipir.gr" ns1:OutboundIP="true"/>
     <ns1:ProcessorLoad ns1:Last15Min="45" ns1:Last1Min="337"
     ns1:Last5Min="126"/>
    </ns1:Host>
   </ns1:SubCluster>
  </ns1:Cluster>
 </ns1:GLUECE>
\end{lstlisting}

\subsubsection{XSLT}
\newpage

\subsubsection{WebMDS and XPath}
query examples

interface and forms description
\newpage
XPath query:
\begin{verbatim}
//glue:Host[@glue:Name='ltsp.oslab.teipir.gr']
\end{verbatim}
Result:
\begin{lstlisting}[language=XML,caption=WebMDS results from XPath query]
<WebmdsResults>
	<ns1:Host ns1:Name="ltsp.oslab.teipir.gr" ns1:UniqueID="ltsp.oslab.teipir.gr">
		<ns1:Processor ns1:CacheL1="0" ns1:CacheL1D="0" ns1:CacheL1I="0" ns1:CacheL2="0" ns1:ClockSpeed="1600" ns1:InstructionSet="x86_64"/>
		<ns1:MainMemory ns1:RAMAvailable="17806" ns1:RAMSize="20121" ns1:VirtualAvailable="22137" ns1:VirtualSize="24508"/>
		<ns1:OperatingSystem ns1:Name="Linux" ns1:Release="2.6.32-24-server"/>
		<ns1:Architecture ns1:SMPSize="8"/>
		<ns1:FileSystem ns1:AvailableSpace="34243" ns1:Name="entire-system" ns1:ReadOnly="false" ns1:Root="/" ns1:Size="251687"/>
		<ns1:NetworkAdapter ns1:IPAddress="192.168.0.101" ns1:InboundIP="true" ns1:MTU="0" ns1:Name="ltsp.oslab.teipir.gr" ns1:OutboundIP="true"/>
		<ns1:ProcessorLoad ns1:Last15Min="9" ns1:Last1Min="1" ns1:Last5Min="9"/>
	</ns1:Host>
</WebmdsResults>
\end{lstlisting}
raw XML output from WebMDS
\newpage

\subsection{BDII}
/opt/glite/etc/gip/provider/glite-info-provider-service-xxx
create a wrapper to present ganglia url and status to the BDII

Ganglia official python client:
\begin{lstlisting}[language=bash,caption=Python Ganglia client MDS export]
[root@mon ~]# /opt/ganglia/bin/ganglia --format=MDS | grep -A 30 host=gr03

dn: host=gr03.oslab.teipir.gr, scl=sub2, cl=datatag-CNAF, \
 mds-vo-name=local, o=grid
objectclass: GlueHost
GlueHostName: gr03.oslab.teipir.gr
GlueHostUniqueID: RDLAB-TEIPIR-gr03.oslab.teipir.gr
objectclass: GlueHostArchitecture
GlueHostArchitecturePlatformType: x86-Linux
GlueHostArchitectureSMPSize: 2
objectclass: GlueHostProcessor
GlueHostProcessorClockSpeed: 2392
objectclass: GlueHostMainMemory
GlueHostMainMemoryRAMSize: 1035104
GlueHostMainMemoryRAMAvailable: 306280
objectclass: GlueHostNetworkAdapter
GlueHostNetworkAdapterName: gr03.oslab.teipir.gr
GlueHostNetworkAdapterIPAddress: 10.0.0.33
GlueHostNetworkAdapterMTU: unknown
GlueHostNetworkAdapterOutboundIP: 1
GlueHostNetworkAdapterInboundIP: 1
objectclass: GlueHostProcessorLoad
GlueHostProcessorLoadLast1Min: 2.57
GlueHostProcessorLoadLast5Min: 1.48
GlueHostProcessorLoadLast15Min: 0.58
objectclass: GlueHostSMPLoad
GlueHostSMPLoadLast1Min: 2.57
GlueHostSMPLoadLast5Min: 1.48
GlueHostSMPLoadLast15Min: 0.58
objectclass: GlueHostStorageDevice
GlueHostStorageDeviceSize: 209555000
GlueHostStorageDeviceAvailableSpace: 197120000
GlueHostStorageDeviceType: disk
\end{lstlisting}

Perl script to export MDS format:
\begin{lstlisting}[language=bash,caption=Perl Ganglia Information Provider for MDS]
[root@mon ~]# ./ganglia_ip -h mon -p 8649 -o mds | grep -A 22 host=gr03

dn: host=gr03.oslab.teipir.gr, cl=RDLAB, \
 mds-vo-name=local, o=grid
objectclass: GlueHost
GlueHostName: gr03.oslab.teipir.gr
GlueHostUniqueID: RDLAB-TEIPIR-gr03.oslab.teipir.gr
objectclass: GlueHostProcessorLoad
GlueHostProcessorLoadLast1Min: 2.57
GlueHostProcessorLoadLast5Min: 1.48
GlueHostProcessorLoadLast15Min: 0.58
objectclass: GlueHostSMPLoad
GlueHostSMPLoadLast1Min: 2.57
GlueHostSMPLoadLast5Min: 1.48
GlueHostSMPLoadLast15Min: 0.58
objectclass: GlueHostArchitecture
GlueHostArchitectureSMPSize: 2
objectclass: GlueHostProcessor
GlueHostProcessorClockSpeed: 2392
objectclass: GlueHostNetworkAdapter
GlueHostNetworkAdapterName: gr03.oslab.teipir.gr
GlueHostNetworkAdapterIPAddress: 10.0.0.33
objectclass: GlueHostMainMemory
GlueHostMainMemoryRAMSize: 1035104
GlueHostMainMemoryRAMAvailable: 306280
\end{lstlisting}

\begin{lstlisting}[language=bash,caption=BDII LDAP search for Glue CE ProcessorLoad attributes]

# ldapsearch -H ldap://osweb.teipir.gr:2170 -x \
-b GlueHostName=ainex.local,Mds-Vo-name=local,o=grid \
GlueHostProcessorLoadLast1Min GlueHostProcessorLoadLast5Min \
GlueHostProcessorLoadLast15Min

# ainex.local, local, grid
dn: GlueHostName=ainex.local,Mds-Vo-name=local,o=grid
GlueHostProcessorLoadLast1Min: 27
GlueHostProcessorLoadLast15Min: 22
GlueHostProcessorLoadLast5Min: 20

\end{lstlisting}

\section{Description of Information}

\subsection{DOM}

\begin{lstlisting}[language=PHP,caption=PHP DOM call to WebMDS]
$url="http://osweb.teipir.gr:8080/webmds/webmds?info=indexinfo&xsl=&xmlSource.indexinfo.param.xpathQuery=%2F%2F*[local-name%28%29%3D%27Host%27]";
$file  = file_get_contents($url);
$dom = DOMDocument::loadXML($file);
$host = $dom->getElementsByTagName('Host');
$procload = $host->item($k)->getElementsByTagName('ProcessorLoad');
echo $procload->item($i)->getAttribute('Last1Min');
\end{lstlisting}

\subsection{LDAP}

\begin{lstlisting}[language=PHP,caption=PHP LDAP call to BDII]
$ds=ldap_connect("osweb.teipir.gr","2170");
if ($ds)
{
    $r=ldap_bind($ds);
    $sr=ldap_search($ds, "mds-vo-name=local,o=grid", "(&(objectClass=GlueHostProcessorLoad))");
    if ($sr)
    {
         $info = ldap_get_entries($ds, $sr) or die("could not fetch entries");
         echo ($info[0][gluehostprocessorloadlast1min][0])/100;
    }
ldap_close($ds);
}
\end{lstlisting}
\newpage
\subsection{Nagios}
\begin{table}[ht]
\small\addtolength{\tabcolsep}{-3pt}
\begin{tabular}{ | l | l | l | l | l |}
\hline
 Host & Service & Status & Last Check & Status Information \\ \hline
 gr129 & load\_fifteen & \colorbox{green}{OK} & 02-08-2011 20:17:23 & CHECKGANGLIA OK: load\_fifteen is 0.17 \\
\hline
  & load\_five & \colorbox{green}{OK} & 02-08-2011 20:18:12 & CHECKGANGLIA OK: load\_five is 0.27 \\
\hline
  & load\_one & \colorbox{green}{OK} & 02-08-2011 20:17:43 & CHECKGANGLIA OK: load\_one is 0.02 \\
\hline
 gr130 & load\_fifteen & \colorbox{green}{OK} & 02-08-2011 20:14:23 & CHECKGANGLIA OK: load\_fifteen is 1.77 \\
\hline
  & load\_five & \colorbox{yellow}{WARNING} & 02-08-2011 20:14:15 & CHECKGANGLIA OK: load\_five is 4.75 \\
\hline
 & load\_one & \colorbox{red}{CRITICAL} & 02-08-2011 20:14:43 & CHECKGANGLIA OK: load\_one is 11.60 \\
\hline
\end{tabular}
\caption{Example Nagios service status details for ganglia check}
\label{tab:nagios_service_detail}
\end{table}

screenshot of output (green/yellow/red) colored

\newpage

pages from \url{http://people.brunel.ac.uk/~dc09ttp}

\newpage
