% chapter Results

\section{Events source}

Results are examined on the generation of metrics, during the aggregation using various information services and they are presented using ready and custom developed interfaces.

\subsection{Unix stuff}

As described in subsection \nameref{subsec:metrics} of the previous chapter, Linux provides through the {\bf proc pseudo-filesystem} a simple file interface to the metrics taken from the scheduler of processes that are queued in the processor.

The three metrics about CPU load on 1, 5 and 15 minutes average are displayed as follow:

\begin{verbatim}
[root@gr03 ~]# cat /proc/loadavg 
2.29 0.73 0.32 1/230 3584
\end{verbatim}

which may also be displayed using the {\bf uptime command}:

\begin{lstlisting}
[root@gr03 ~]# uptime
 00:01:20 up  1:41,  3 users, load average: 2.29, 0.73, 0.32
\end{lstlisting}

When we examine the Linux kernel source code, there is a macro command named $CALC_LOAD$ which takes the options that have been discussed and returns the result of the metric. The definition of the macro can be seen in file {\bf include/linux/sched.h}, Listing \ref{kernel}.

\subsection{Ganglia}

When gmond starts, it listens on port 8649/TCP by default, to accept TCP connections and throw XML report for the whole cluster. It also binds to the multicast address on port 8649/UDP to get other hosts messages for metrics changes, and also multicast its own metrics. Listing \ref{lsof} shows the opened sockets of Gmond daemon, and Listing \ref{telnet_gmond} display a sample xml output when connecting to 8649/TCP to transfer metrics through XML.

Worker nodes are configured to transfer metrics data using multicast. Each Gmond daemon of each Computing Element node, by Ganglia definition has to know the state of the whole Computing Element cluster. Using standard UNIX commands to listen to the data transferred on the multicast network, a sample transfer of load\_one metric is observed (Listing \ref{tcpdump}). As described in Subsection \ref{subsec:ganglia}, metric data are multicasted by Gmond when there is a change in the value, or when the time threshold is reached.

Using Ganglia build-in command $gstat$, a nice output of Processor Load metrics is shown for the whole cluster in Listing \ref{gstat}

\begin{lstlisting}[language=bash,caption=Gstat output,label=gstat]
[root@gr01 ~]# gstat -al1
gr03.oslab.teipir.gr     2 (    0/   87) [  0.00,  0.00,  0.00] [   0.0,   0.0,   0.0,  99.9,   0.1]
gr01.oslab.teipir.gr     1 (    0/   75) [  0.00,  0.00,  0.00] [   0.0,   0.0,   0.0,  99.9,   0.0]
gr02.oslab.teipir.gr     1 (    0/   99) [  0.00,  0.00,  0.00] [   0.0,   0.0,   0.1,  99.9,   0.0]
\end{lstlisting}

\section{Aggregation and transfer}

Metrics taken from the event source are passed to the information service using the information/resource provider.

\subsection{WSRF}

In WSRF the $wsrf-query$ command is executed using the URL of the container where the WSRF was been deployed. The container should run in a host with a host certificate signed by a certificate authority, and a certificate proxy should be initialized with a valid user certificate that is requesting the information. 

\begin{lstlisting}
/opt/globus/bin/wsrf-query -s https://osweb.teipir.gr:8443/wsrf/services/DefaultIndexService "//*[local-name()='Host']"
\end{lstlisting}

The result of the above query returns a list of all cluster nodes. An abstract result is displayed in Listing \ref{wsrfquery}.

\subsubsection{WebMDS and XPath}

The role-based access control model of the grid security context allow queries only by authenticated and authorized users. Building a testbed with full grid security in mind would be out of this project scope, so XML result of Web Service calls is taken through WebMDS instead of \ac{WSDL} discovery and \ac{SOAP} messaging.

Using the following XPath query in the WebMDS form, a request to get the metrics of Host node with name $ltsp.oslab.teipir.gr$ was sent.

\begin{verbatim}
//glue:Host[@glue:Name='ltsp.oslab.teipir.gr']
\end{verbatim}

WebMDS match the query in its cache and replies only the specific node that was requested. If the cache was expired (its default value is 60 seconds) the information system uses the Resource Provider to fetch the XML from Gmond and transform it using XSLT to Glue schema and serve the new values as shown in Listing \ref{xpath_result}.

\subsection{BDII}

On the other information service, using $ldapsearch$ command and specifying the base DN for the search, the host URI and the desired attributes to return, we may get the values asked from the BDII as seen on Listing \ref{ldapsearchbdii}.

\begin{lstlisting}[language=bash,caption=BDII LDAP search for Glue CE ProcessorLoad attributes,label=ldapsearchbdii]
# ldapsearch -H ldap://osweb.teipir.gr:2170 -x \
-b GlueHostName=ainex.local,Mds-Vo-name=local,o=grid \
GlueHostProcessorLoadLast1Min GlueHostProcessorLoadLast5Min \
GlueHostProcessorLoadLast15Min

# ainex.local, local, grid
dn: GlueHostName=ainex.local,Mds-Vo-name=local,o=grid
GlueHostProcessorLoadLast1Min: 27
GlueHostProcessorLoadLast15Min: 22
GlueHostProcessorLoadLast5Min: 20
\end{lstlisting}

The above information has been given by the BDII LDAP instance which used the wrapper of Ganglia Resource Provider, a customized Perl script to export MDS format as shown in Listing \ref{perlbdii}.

Using the Ganglia official python client (Listing \ref{pythonbdii}) which is distributed with the source code of Ganglia Client, rejected as an option because Perl is easier in handling regular expressions for string operations and transformation.

\section{Presentation}

Finally, to present the information, two custom interfaces have been developed in PHP. Both programs reside in Brunel University webserver which supports the needed libraries and network connections. The source code is available under that page, and when called using BDII or WSRF, the result is exactly the same and looks like Table \ref{tab:html_output}.

\begin{table}[ht]
\centering
%\small\addtolength{\tabcolsep}{-3pt}
\begin{tabular}{ | l | l | l | l | l |}
\hline
 Hostname & 1min & 5min & 15min \\ \hline
 ltsp.oslab.teipir.gr & \colorbox{green}{0.01} & \colorbox{green}{0.09} & \colorbox{green}{0.09} \\ \hline
 xenia.oslab.teipir.gr & \colorbox{green}{0.00} & \colorbox{green}{0.06} & \colorbox{green}{0.06} \\ \hline
 gr201.oslab.teipir.gr & \colorbox{green}{0.52} & \colorbox{green}{0.59} & \colorbox{green}{0.42} \\ \hline
 gr130.oslab.teipir.gr & \colorbox{green}{0.00} & \colorbox{green}{0.06} & \colorbox{green}{0.16} \\ \hline
 gr131.oslab.teipir.gr & \colorbox{yellow}{1.22} & \colorbox{green}{0.79} & \colorbox{green}{0.40} \\ \hline
 gr212.oslab.teipir.gr & \colorbox{green}{0.06} & \colorbox{green}{0.09} & \colorbox{green}{0.09} \\ \hline
 gr180.oslab.teipir.gr & \colorbox{red}{2.06} & \colorbox{yellow}{1.49} & \colorbox{green}{0.59} \\ \hline
 gr181.oslab.teipir.gr & \colorbox{green}{0.71} & \colorbox{green}{0.57} & \colorbox{green}{0.32} \\ \hline
\end{tabular}
\caption{Sample output from both calls with DOM or LDAP}
\label{tab:html_output}
\end{table}

Page \url{http://people.brunel.ac.uk/~dc09ttp} contains complete results using both DOM and LDAP methods.

\subsection{DOM}

The first interface uses PHP \ac{DOM} functions to call the WebMDS and get the XML as it would happened by calling the WSRF Web Service with native SOAP messages and authentication. Listing \ref{wsrf.php} is an abstraction of the code deployed in Brunel's webserver and returns only one value. The purpose of this listing is to present the use of functions and not the HTML stuff.

\begin{lstlisting}[language=PHP,caption=PHP DOM call to WebMDS,label=wsrf.php]
$url="http://osweb.teipir.gr:8080/webmds/webmds?info=indexinfo&xsl=&xmlSource.indexinfo.param.xpathQuery=%2F%2F*[local-name%28%29%3D%27Host%27]";
$file  = file_get_contents($url);
$dom = DOMDocument::loadXML($file);
$host = $dom->getElementsByTagName('Host');
$procload = $host->item($k)->getElementsByTagName('ProcessorLoad');
echo ($procload->item($i)->getAttribute('Last1Min'))/100;
\end{lstlisting}

\subsection{LDAP}

The second interface uses LDAP functions to connect to BDII instance and get the results from objects that instantiates the GlueHostProcessorLoad class. Listing \ref{bdii.php} displays the method to bind anonymously, form the query and get a sample result.

\begin{lstlisting}[language=PHP,caption=PHP LDAP call to BDII,label=bdii.php]
$ds=ldap_connect("osweb.teipir.gr","2170");
if ($ds)
{
    $r=ldap_bind($ds);
    $sr=ldap_search($ds, "mds-vo-name=local,o=grid", "(&(objectClass=GlueHostProcessorLoad))");
    if ($sr)
    {
         $info = ldap_get_entries($ds, $sr) or die("could not fetch entries");
         echo ($info[0][gluehostprocessorloadlast1min][0])/100;
    }
ldap_close($ds);
}
\end{lstlisting}

\subsection{Nagios}

Nagios calls check\_ganglia periodically (configured as check interval) and logs the state of each service and host check. Load averages are described as services. Listing \ref{nagioslog} shows the log file of nagios service check for these values straight from Ganglia.

\ac{NPCD} is configured in bulk mode so a file may be found in the spool directory before the interval passes to process it with perl script to \ac{RRD} database. Listing \ref{npcdfile} contains the performance data metrics.

Finally, Nagios Web interface displays the aggregated values of host performance state in multiple views, as shown in Table \ref{tab:nagios_service_detail}.

\begin{table}[ht]
\small\addtolength{\tabcolsep}{-3pt}
\scalebox{0.8}{
\begin{tabular}{ | l | l | l | l | l |}
\hline
 Host & Service & Status & Last Check & Status Information \\ \hline
 gr129 & load\_fifteen & \colorbox{green}{OK} & 02-08-2011 20:17:23 & CHECKGANGLIA OK: load\_fifteen is 0.17 \\
\hline
  & load\_five & \colorbox{green}{OK} & 02-08-2011 20:18:12 & CHECKGANGLIA OK: load\_five is 0.27 \\
\hline
  & load\_one & \colorbox{green}{OK} & 02-08-2011 20:17:43 & CHECKGANGLIA OK: load\_one is 0.02 \\
\hline
 gr130 & load\_fifteen & \colorbox{green}{OK} & 02-08-2011 20:14:23 & CHECKGANGLIA OK: load\_fifteen is 1.77 \\
\hline
  & load\_five & \colorbox{yellow}{WARNING} & 02-08-2011 20:14:15 & CHECKGANGLIA OK: load\_five is 4.75 \\
\hline
 & load\_one & \colorbox{red}{CRITICAL} & 02-08-2011 20:14:43 & CHECKGANGLIA OK: load\_one is 11.60 \\
\hline
\end{tabular}}
\caption{Example Nagios service status details for ganglia check}
\label{tab:nagios_service_detail}
\end{table}

